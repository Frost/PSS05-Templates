\def\thedocument{User Requirements Document}
\def\thedate{DATE} % TODO: insert proper date here
\def\theversion{0.1}
\def\thestatus{Working copy}

\documentclass[notitlepage,headsepline,twoside,a4paper,11pt]{report}
\usepackage[english]{babel}
% Uncomment this line if not using XeTeX and you need unicode characters
% \usepackage[utf8]{inputenc}
\usepackage[T1]{fontenc}
\usepackage{listings}
\usepackage{tabularx}
\usepackage[top=2.5cm,bottom=2.5cm,nohead,nofoot]{geometry}
\usepackage{pdfpages}
\usepackage{fancyhdr}
\usepackage{ifthen}
\usepackage{color}
\definecolor{dark-blue}{rgb}{0, 0, 0.6}
\usepackage{hyperref}
\hypersetup{
  pdfpagemode=FullScreen,
  colorlinks=true, 
  linkcolor=dark-blue,
  urlcolor=dark-blue
}

% fulhack för att se till att vi inte får med kapitel-numrering.
\renewcommand{\chaptername}{}
\renewcommand{\thechapter}{}
\renewcommand{\thesection}{\arabic{section}}
\renewcommand{\and}{\\}

\newcommand{\documentstatus}{\small{\theversion\ -- \thestatus}}

\newenvironment{itemize*}{
\begin{itemize}
  \setlength{\itemsep}{1pt}
  \setlength{\parskip}{0pt}
  \setlength{\parsep}{0pt}
}{\end{itemize}}

\def\requirement#1#2{
  \subsubsection*{}
  \noindent
  \setlength{\extrarowheight}{4pt}
  \begin{tabularx}{\linewidth}%
    {l>{\setlength\hsize{0.67\hsize}}X% 
    >{\setlength\hsize{1.33\hsize}}X} 
     \toprule

  % Table content
  \multicolumn{2}{c}{\textbf{#1}} \\ \midrule
  #2
   \bottomrule 
  \end{tabularx} \\ \\ \\
}

\pagestyle{fancy}
\headheight 14pt
\fancyfoot{}
\lhead{\thecode\ \thecourse}
\rhead{\theproject}
\fancyfoot[LE,RO]{\thepage}
\setlength\footskip{9.6pt}

\def\thecourse{COURSE NAME}   % TODO: Enter course name
\def\thecode{XX1234}          % TODO: Enter course id
\def\theproject{PROJECT NAME} % TODO: Enter project name

\begin{document}
\fancypagestyle{plain}
{
    \fancyhead{}
    \fancyfoot{}
    \fancyfoot{}
    \lhead{\thecode\ \thecourse}
    \rhead{\theproject}
    \fancyfoot[LE,RO]{\thepage}
} % clear header and footer of plain page because of ToC

\fancypagestyle{emptyfoot}{
  \fancyfoot{}
}
\begin{titlepage}
  \title{\thedocument\\\theproject\\\documentstatus}
  \author{% Authors of this Document
% TODO: insert authors here
Foo Bar\and
Foo Bar\and
Foo Bar\and
Föö Bär}
  \date{\thedate} % TODO: insert date here
\end{titlepage}
\maketitle
\thispagestyle{emptyfoot}
\setcounter{secnumdepth}{3}
\setcounter{tocdepth}{3}
\newpage
\thispagestyle{emptyfoot}
\begin{abstract}
  % Abstract for URD
\end{abstract}
\tableofcontents
\pagenumbering{roman}
\newpage
% Make sure the next content page starts on a right-hand page.
\cleardoublepage

\chapter*{\thedocument}
\addcontentsline{toc}{chapter}{\thedocument}
\pagenumbering{arabic}
\setcounter{page}{1}

\section{Introduction} % (fold)
\label{sec:introduction}
\subsection{Document Status} % (fold)
\label{sub:document_status}
\documentstatus
% subsection document_status (end)
% A record of changes between different versions of the document
% TODO: Add versions here if you're supposed to hand in an updated version of the document.
\subsection{Document Change Record} % (fold)
\label{sub:document_change_record}
\begin{itemize}
  \item \theversion
\end{itemize}
% subsection document_change_record (end)

% section introduction (end)

% URD Table of Contents
% Introduction. This chapter should give an overview of the whole document
\section{Introduction}
  \subsection{Purpose}
    % Purpose of this document

  \subsection{Scope of the software}
    % Scope of the software. 
% An "executive summary" of the product under development.
% Not more than 30-40 words.

  \subsection{Definitions, acronyms and abbreviations}
    % Definitions
\subsubsection*{Technical} 
\begin{itemize}
	\item Foo -- Bar
\end{itemize}


  \subsection{References}
    % References. 
% Sources of additional information helpful in reading this document, 
% with a brief explanation of the contents and usefulness of each. 
% Could be customers in-house reports, reports from previous projects, 
% scientific or technical reports, industry white papers, computer science or other books, 
% on-line references (URLs) and related web sites, newspaper articles.
% You could for example make use of the \hyperref command
\begin{itemize}
\item Foo -- Bar
\end{itemize}


  \subsection{Overview of the Document}
    % Overview of the document



% General Requirements
\section{General Requirements}
  \subsection{Product Perspective}
    % Product perspective. 
% Describes related external systems and subsystems. How the product under development fits into an existing environment.


  \subsection{General Capabilities}
    % General Capabilities. 
% Describes the main capabilities required and why they are needed from the end users perspective. 
% Gives an overview of the product under development. 
% Takes a user-centric approach.


  \subsection{General Constraints}
    % General Constraints


  \subsection{User Characteristics}
    % User characteristics. 
% Describes who will use the software, expected background, previous training and level of skill (may be several). 
% Identify different job roles and contexts of use (can be used to develop a use case analysis). 
% Used to determine user interface requirements, online/offline user support and product documentation.

\subsubsection{Some user}
Foo.
\subsubsection{Some other user}
Bar.
\subsubsection{Some third user}
Baz.


  \subsection{Assumptions and Dependencies}
    % Assumptions and dependencies. 
% Describes the assumptions upon which the requirements depend. 
% For example technical assumptions on levels of hardware performance, system availability and reliability. 
% May include commercial assumptions about customers business needs/ business model/ business process.


  \subsection{Operational environment}
    % Operational environment. 
% Describes what any external systems do, in terms of functionality, (e.g. file servers, databases, etc) Describes the interfaces made available to the product under development.


% Specific Requirements 
% Provides a detailed list of specific requirements organised into two categories.
\section{Specific Requirements}
  \subsection{Capability requirements}
    % Capability requirements. 
% Functional requirements on the system, (what the system should actually do) including interface requirements (e.g. file formats, data definitions, APIs, communications protocols etc.) in so far as these are known at an early stage.

\subsection{Capability requirements}

\subsubsection{some capability subsection}

\requirement{Foo Capability}{
  Requirement Description &  \\ 
  Justification &  \\ 
  Need &  \\
  Priority &  \\
  Stability &  \\
  Source &  \\
  Verifiability & \\
}

\requirement{Bar Capability}{
  Requirement Description &  \\ 
  Justification &  \\ 
  Need &  \\
  Priority &  \\
  Stability &  \\
  Source &  \\
  Verifiability & \\
}


  \subsection{Constraint requirements}
    % Constraint requirements. 
% Non-functional requirements, including:
% * performance requirements (speed of execution, memory requirements),
% * environment requirements (hardware, OS, peripherals, network, web browser)
% * external requirements (minimum version numbers of external systems, subsystems, functionality needed from these)
% * reliability requirements (uptime, mean time to failure, accessibility, loading, average performance, worst case performance, etc)
% * usability requirements (minimum time to learn system, expected time to learn system, level of user support, expected efficiency gains from system)
% * safety requirements (critical functionality and its reliability)
% * legal requirements (conformance to industry standards and recommendations)

\subsubsection{Some subsection to constraint requirements}

\subsubsection{some capability subsection}

\requirement{Foo Constraint}{
  Requirement Description &  \\ 
  Justification &  \\ 
  Need &  \\
  Priority &  \\
  Stability &  \\
  Source &  \\
  Verifiability & \\
}

\requirement{Bar Constraint}{
  Requirement Description &  \\ 
  Justification &  \\ 
  Need &  \\
  Priority &  \\
  Stability &  \\
  Source &  \\
  Verifiability & \\
}



\appendix

\chapter{Minutes}
	Minutes from meetings at the following dates are included in this appendix:

\input{appendices/minutes/minutes_list}

\include{appendices/minutes/minutes_pdf}


\end{document}

